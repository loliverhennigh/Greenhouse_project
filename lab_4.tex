\documentclass[a4paper,12pt]{article}
%\usepackage{verbatim}


\usepackage{amssymb}
\usepackage{amsmath}
\usepackage{amsthm}
\usepackage{tikz}
\usepackage{enumerate}
\usepackage{mathrsfs}


\usetikzlibrary{arrows,snakes,backgrounds,positioning,shadows}

\tikzstyle{stan} = [circle,draw=red!50,fill=red!20,thin]

\begin{document}

\title{Heat Lab}
\author{Oliver Hennigh \\
Clarkson University}
\renewcommand{\today}{}

\maketitle

\newtheorem*{t1}{Theorem}
\newtheorem*{t3}{Lemma}
\newtheorem*{t2}{Definition}

\begin{abstract}
We have examined properties of converting electrical energy to heat. By setting up a resistor circuit and a calorimeter we were able to capture the energy released from the electrical power magic force. By measuring this with respect to time we could calculate several things about the system. Our recorded data matched gave us good predicted values of several constants.
\end{abstract}


\section{Method}
If there was not much to do in lab three then there really want much to do in this one. We had to measure the mass of the water by first measuring the empty calorimeter and then both the calorimeter and water. The water at this point was in the calorimeter. This poses the problem of how to find the mass of the water. We we devised a method to do so. Using the simple and elegant power of addition and subtraction we notice something very strange. Letting $m_w$ be mass of water, $m_c$ be mass of calorimeter, and $m_{cw}$ be the measured mass of both,

\begin{equation}
	m_{cw} = m_w + m_c
\end{equation}

Wow, and they say there are no more heroes. Now we can use this to get the mass of the water

\begin{equation}
	m_w = m_{cw} - m_c
\end{equation}

Well Okie dokie then. Now we need to find like $J$ or something. We can use this equation from the lab book

\begin{equation}
	S = \frac{IV}{J(mC_w + M_cC_c + M_sC_s)}
\end{equation}

Where $S$ is the slope of the line temperature vs time. $C_w$ is the specific heat of water J/mC, $C_c$ the specific heat of the calorimeter J/mC, $C_s$ the specific heat of the stirring rod J/mC, and $M_s$ is the mass of said rod in grams. Those are the working equations of the lab. 

\section{Data and Results}

\subsection{Temperature readings over time}

These measurements were gathered while performing the experiment. Measuring the masses of each component we found $m_l = 44.7 g$, $m_w = 210.9 g$. We already know that $m_sC_s = 2.5$ cal/C and $C_l = 0.21$ cal/gC. Water is of course $C_w = 1$ cal/gC.

\begin{table}[ht]
\caption{temp measurements while resistor heating}
\centering
\begin{tabular} {c c c c}
\hline\hline
 Temp in C & time in sec & V in Volts & I in amps \\ [0.5ex]
\hline
15.1 & 0 & 6.8 & 1.5 \\
16.3 & 92 & 6.8 & 1.5 \\
17.1 & 176 & 6.8 & 1.5 \\
18 & 255 & 6.8 & 1.5 \\
19 & 355 & 6.8 & 1.5 \\
20 & 449 & 6.8 & 1.5 \\
21 & 575 & 6.8 & 1.5 \\
22 & 655 & 6.8 & 1.5 \\
23 & 758 & 6.8 & 1.5 \\
24 & 848 & 6.8 & 1.5 \\
25 & 935 & 6.8 & 1.5 \\
26 & 1069 & 6.8 & 1.5 \\
27 & 1170 & 6.8 & 1.5 \\
28 & 1258 & 6.8 & 1.5 \\
29 & 1338 & 6.8 & 1.5 \\
30 & 1453 & 6.8 & 1.5 \\
31 & 1530 & 6.8 & 1.5 \\
32 & 1644 & 6.8 & 1.5 \\
\hline
\end{tabular}
\end{table}


\subsection{Question 1}
Plot temp vs time and average $V$ as well as $I$ to determine $J$

Plot on back. We calculated the average values if $V = 6.8 Volts$ and $I = 1.5 amps$. Using this slope = .0101 and the equation

\begin{equation}
S = \frac{IV}{J(mC_w + M_cC_c + M_sC_s)}
\end{equation}

We find the value of $J = \frac{1010}{mC_w + M_cC_c + M_sC_s} = 4.55$ joules/ cal. Thats pretty close to the $4.19$ joules/ cal that it really is.

\subsection{Question 3}
Will a heating coil that has lower resistance take longer of shorter to heat up

The power dissipation is $P = \frac{V^2}{R}$ so given the same voltage and lower resistance $P$ will increase. This means that the experiment will be shorter as it will take less time to heat up.
\subsection{Question 4}
The power dissipated in a resistor is inversely proportional to the resistance.

True for the same reason in 3. $P = \frac{V^2}{R}$.
\subsection{Question 5}
A 1 kW heater is designed to operate at 220 V. (a) What is its resistance, and what current does it draw? (b) What is the power of this resistor if it operates at 110V?

(a) Well $P = \frac{V^2}{R}$ so $R = 48.4$ Ohms. $V/R = I$ so $I = 4.54$ amps. 

(b) Then $P = \frac{V^2}{R}$ so $P = 250$ Watts.
\subsection{Question 7}
coper bowl 150g. Water 220g. Temperature 20.0 C. Copper cylinder 300 g. Steam 5 g. Final temp 100 C.

(a) Well the water went from 20 to 100 C so given specific heat 1 cal/g (C) and 220 grams this gives $H_w = 1 * 220 * 80 = 17600$ cal. 

(b) Well the bowl went from 20 to 100 C so given specific heat .0923 cal/ g C and 150 g bowl this gives $H_b = .0923 * 150 * 80 = 1107$ cal.

(c) the energy must be constant so $.0923 * 300 * (t - 100) = 1107 + 17600 + H_v = 18707 + 5 * 540 = 21407$. So $t = 873.09 C$. Seems Ok. 


\subsection{Question 8}
A 1500 kg Buick moving at 90 km/h brakes to stop, at uniform deceleration and without skidding, over a distance of 80 m. At what average rate is mechanical energy transferred to thermal energy in the brake system.

Assuming a uniform deceleration we solve for the acceleration $v_o^2 = 2a(x - x_0)$ so $90^2 / -(80*2) = -50.625 m/s^2$. This gives the time till stop $t = 1.77 sec$. The total energy transferred to thermal energy is $E = 1/2 m v^2 = 1/2 * 1500 * (90 * 1000 / (3600))^2 = 468750 J$. This makes the average dissipation $264830$ W.


\subsection{Question 9}
surface area required to raise the temperature of 200 L water from 20 C to 40 C in 1 h. 

First I will see how many joules it takes to raise the temp of this water. $E = m K (t - t_0) = (200*1000) g * (4.17) J/g C (20) C = 16680000 J$. This needs a wattage of $4633$ W if you need to do it in $3600$ seconds. So the energy produced by the light will be $.2 * 700 = 140 W/m^2$. $4433 / 140 = 31.6 m^2$ and thats how much is needed.

\subsection{Question 10}
What mass of steam must be mixed with 150 g of ice to produce liquid water at $50 C$?

The energy needed to melt the ice will be $150g  *333.5 J/g = 49950 J$. The energy needed to heat it from $0$ C to 50 C is $150 * 4.186 * 50 = 31350 J$. The energy in the steam equal to these things to $49950 + 31350 = m * 4.186 * 50 + m * 2260 = m(4.186 * 50 + 2260)$. So working this out it will take $32.9$ grams of steam.


\section{Discussion}
We have examined properties of heating water. By observing trend lines in temp increase we could determine things like the conversation constant $J$ from joules to cal. I know that there was a correction on the lab that told us to find $\alpha$ instead of $J$. I was not sure what this could be. I suppose I forgot to take notes on it or something.


\end{document}
